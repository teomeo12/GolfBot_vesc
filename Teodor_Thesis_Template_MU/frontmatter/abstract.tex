\chapter*{Abstract}
\addcontentsline{toc}{chapter}{Abstract} % add it to table of contents


    \paragraph{} 
    Maintaining the pristine condition of golf course turf is a significant operational challenge, demanding constant, labor-intensive efforts to repair surface damage such as divots. This manual process is not only costly and time-consuming but can also lead to inconsistent repairs, impacting the quality of play. The advancement of autonomous systems presents a compelling opportunity to automate such specialized maintenance tasks, promising increased efficiency, precision, and a reduction in operational costs.
    
    \paragraph{}
    This project presents the design, implementation, and evaluation of GolfBot, a functional prototype of an autonomous rover created to address the specific task of divot repair. The system is built upon the Wumpus mobile robot platform and enhanced with a suite of specialized hardware and software. Its core functionalities are driven by a computer vision system employing a custom-trained YOLO model for real-time divot detection, an RTK-GPS module for centimeter-level navigation accuracy, and a custom-designed mechanical dispenser for the precise application of a sand-seed mixture. The entire system is integrated and controlled using the Robot Operating System 2 (ROS 2), with key components developed in Python and C++, while a custom GUI provides tele-operation and mission monitoring.

    \paragraph{} 
    The result of this project is a functional prototype that successfully demonstrates an autonomous 'detect and repair' pipeline. In a controlled environment using simulated turf, the system is capable of identifying divots, autonomously aligning itself over the target, and performing precise, automated repairs. This thesis documents the complete development lifecycle, from the initial hardware and software architecture design to the implementation of the core operational logic and initial functional testing. The evaluation of the prototype demonstrates the viability of using autonomous robotics to improve the efficiency and quality of golf course maintenance, establishing a solid foundation for future enhancements and potential commercial application. GolfBot detected divots with an mAP$_{50}$ of 0.954.
     
