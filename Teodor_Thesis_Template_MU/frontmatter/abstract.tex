\chapter*{Abstract}
\addcontentsline{toc}{chapter}{Abstract} % add it to table of contents

Maintaining high-quality golf courses requires regular divot repairs, which is a slow, repetitive, and tiring job for maintenance workers. The GolfBot project proposes an autonomous robot to identify and fix divots on golf courses, addressing the need for a more efficient and automated solution in turf management.

This project, GolfBot, is an autonomous vehicle built upon the Wumpus platform, enhanced with specialized hardware and software to perform its task. The robot uses advanced computer vision techniques, including a custom-trained object detection model for real-time divot identification and a 3D depth camera for accurately measuring divot dimensions. For navigation, it is equipped with RTK GPS modules for precise positioning. The mechanical repair mechanism consists of a hopper with a motor-screw mechanism for dispensing a sand-seed mixture, and rubber brushes for uniform spreading. The entire system is controlled using ROS2, with components developed in Python and C++.

The result of this project is a functional prototype of an autonomous divot-repairing robot. This thesis documents the design, implementation, and evaluation of the GolfBot, demonstrating its capability to improve the quality and efficiency of golf course maintenance.
