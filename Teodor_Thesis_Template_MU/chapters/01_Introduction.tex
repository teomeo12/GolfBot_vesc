\chapter{Introduction}
\label{chap:introduction}

You can cite like this~\citep{DiCarlo_2018_DynamicLocomotion} for number only or you can cite like this for the author:~\citet{DiCarlo_2018_DynamicLocomotion}.


\section{Motivation}
% The motivation for the project.

\section{Problem Statement}
% The problem statement of the project.


\section{The GolfBot System}
% Explain what the golf bot's purpose is that is built on the Wumpus platform (describe the hardware). 
% Then explain how Wumpus is enhanced with the additional hardware (camera, Jetson, NEMA motor, Arduino, etc.) that you can see in Figure 1. 
% Describe the software part ROS2 , python , Arduino, YOLO.

\section{Thesis Outline}
% In chapter one I'm going to talk about this. 
% In chapter 2, I'm going to talk about this. 
% In chapter 3, I'm going to talk about this.
% The document is split into a few parts or the main components that make this robot operate. These are : Computer Vision system, Navigation, Mechanical Dispenser, GUI
The following chapters will detail the project.
\begin{itemize}
    \item \textbf{Chapter 2: Literature Review} will survey existing work on autonomous robots in agriculture and turf maintenance, divot detection techniques, and automated repair systems.
    \item \textbf{Chapter 3: Design} will present the overall design of the GolfBot, covering both the hardware and software architectures.
    \item \textbf{Chapter 4: Implementation} will provide a detailed account of the implementation of the GolfBot's main components: the computer vision system, the navigation system, the mechanical dispenser, and the graphical user interface.
    \item \textbf{Chapter 5: Experimental Results} will describe the experiments conducted to validate the performance of the GolfBot and present the results.
    \item \textbf{Chapter 6: Future Work} will discuss potential enhancements and future research directions for the GolfBot project.
    \item \textbf{Chapter 7: Conclusion} will summarize the project, highlighting its contributions and concluding remarks.
\end{itemize}